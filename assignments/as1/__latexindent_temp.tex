\documentclass[12pt]{article}
\usepackage[utf8]{inputenc}
\usepackage{fancyhdr}
\usepackage[yyyymmdd]{datetime}

\newcommand\name{Nan Wang}

\pagestyle{fancy}
\fancyhf{}
\chead{\textbf{Note for Lec 8}}
\lhead{\name}
\rhead{\today}
\rfoot{\small \thepage}



\begin{document}

\begin{center}
    \huge\textbf{How is the Actor Regulation Influencing the Movie Market?}
\end{center}

\section{Introduction}
On Mar 16th 2021, the National Radio and Television Administration(NRTA) of China published the first draft of Radio and Television Law, in which actors and celebrities with misconducts or illegal acts, such as taking drugs and prostitution, are banned from any public performing activities. Actually it is not the first time for NRTA to introduce regulations against actors with misconducts. Back in 2014, the NRTA has already introduced regulations limiting the transmit of movies or TV shows containing actors with misconducts.

The introduction of regulations on actors took places as an exogenous shock, making investment in movie industry becomes more risky. So for rational producers, they would invest more on safer movies, the so-called ``Main Melody" Film, with experienced actors with positive social images. Therefore, the shift in investment can lead to a decrease in movie variety.

\section{Research Question}
The introduction of actor regulation provided us with a ideal window to learn about the movie market and consumer's behavior as it exogenously decreases the ``product variety" on the market.

The research questions are:
\begin{itemize}
\item Whether the introduction of actor regulation decreased the movie variety?
\item Do the change in movie variety impact consumption?
\end{itemize}

We can study those two questions by event study or DID setting, using the movie market in Hong Kong, Taiwan or Singapore as control group. As for data, ideally, we can use the selling data from movie ticket apps to see micro and aggregate movie ticket selling.





\end{document}
